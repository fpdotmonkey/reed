%!TEX TS-program = xelatex
%!TEX encoding = UTF-8 Unicode

\documentclass[12pt,american,german,british]{article}
\usepackage{geometry}
\geometry{a4paper}

\usepackage{hyperref}
\usepackage{fontspec,xltxtra,xunicode}
\usepackage{babel}
\usepackage{setspace}
\usepackage{siunitx}
\usepackage{subcaption}
\usepackage{parskip}
\usepackage{rotating}
\usepackage{makecell}

\title{Development of a Robotic Fleet for Agriculture }
\date{\today}

\author{
	Fletcher Porter, me@fletcherporter.com
	\and Gert Kauniste, gert.kauniste@aalto.fi
	\and Oskari Hynninen, oskari.hynninen@aalto.fi
	\and Yelin Hou, yelin.hou@aalto.fi
}

\newcommand{\reedtemp}[1] {
	\textit{\textcolor{blue}{#1}}
}

\begin{document}

\raggedright

\maketitle

\begin{abstract}
\reedtemp{
The abstract will state what problems we seek to solve, the high level of how
we planned to solve them, and how successful it was.  All in a single
paragraph.
}
\end{abstract}

\section{Introduction}

Under the present threat of the climate crisis, it is the important challenge
of our time to develop new technologies to reduce our overall emissions of
greenhouse gasses \cite{IPCC2022}.  One area where such improvements can be
made is in the care of common reeds ({\it Phragmites australis}).  Reeds are a
plant that grow in wetland biomes around the world.  They aren't typically
cultivated for anything.  It's even quite often that they are burned to get
them out of the way of human activity.  This is problematic both for that
burning reeds produces greenhouse gasses and the ash left over from burning
enters the wetland ecosystem, making water and soil more alkaline
\cite{Sluis2013}.

We propose that these wild reeds should be harvested and used as a resource for
energy and raw materials.  Studies by Köbbing et al and van der Sluis et al
have shown that, not only are these applications possible, but they may even be
favorable \cite{Sluis2013}\cite{Koebbing12013}.  There are also environmental
benefits in cutting down reeds in areas where they are an invasive species
\cite{Tulbure2007}.  However, there are technical challenges to overcome for
reed harvesting to become truly viable.

Principally among these challenges is the inherent difficulty of wetland
environments.  The ground is very soft, making driving tractors over it
challenging.  Even if that were to be overcome, these ecosystems are quite
sensitive, so uncareful driving would be to their detriment.  This can be
mitigated by harvesting in the winter when the water is frozen, but care must
still be taken \cite{Sluis2013}.

To overcome this, we propose using many small, lightweight tractors so to
minimize environmental disturbance while getting the reeds quickly.  Employing
many people to drive these small tractors may be economically challenging, so
we imagine them as (at least semi-) autonomous robots.  In one pass several
robots cut down the reeds somewhat like a lawn mower, and in another, robots
pick up the reeds that they harvested.  These robots should be coordinated by a
UAV drone that watches them from above and human operator to provide
high-level instructions.

There's already been a lot of study into applying robotics in agriculture.
Millard et al have proposed, but not implemented, a swarm robot system for
harvesting cereal crops like wheat or rice \cite{Millard2019}.  Blender et al
developed a swarm robot system where robots e.g. plant seeds as they traverse
across a field and deployed a single-robot prototype system \cite{Blender2016}.

Kurita et al deployed an autonomous combine harvester in a rice paddy, a
notably wet environment \cite{Kurita2017}.  Rahman et al developed a means of
plotting a robot trajectory through a field to minimize crop losses
\cite{Rahman2019}.  Akbari et al developed techniques for aerial triangulation,
localization, and mapping, including for agricultural application
\cite{Akbari2021}.  Montoya-Cavero et al have developed a fruit-picking robot
that uses computer vision to identify where to grasp and pull the fruits
\cite{MontoyaCavero2022}.

\section{Methods}

\reedtemp{
A description of the theoretical model of how our reed harvesting system will
work.  It goes through all system components, what they do and why, and why
we architect our system the way we do.
}

\section{Results}

\reedtemp{
The details of implementing our system.  What tools and facilities did we use,
how did we test the system, did we have to deviate from the theoretical model
and why.
}

\section{Discussion}

\reedtemp{
Did it work?  What improvements could be made for future work on this topic?
Could implementation limitations be lifted to allow improvement, or maybe even
an architectural decision was flawed.
}

\bibliography{reed}
\bibliographystyle{ieeetr}

\end{document}


















